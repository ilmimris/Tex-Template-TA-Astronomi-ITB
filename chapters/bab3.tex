\chapter{PENGOLAHAN DATA}
\vspace{1.0cm}

%-----------------------------------------------------------------------------%
\section{METODE 1}
%-----------------------------------------------------------------------------%
	Medan magnet terdeteksi hampir pada setiap objek astrofisika, mulai dari planet hingga pulsar, maupun pada struktur dengan skala yang lebih besar seperti  galaksi. Observasi medan magnet galaksi menunjukkan bahwa keberadaan medan magnet memerankan bagian penting dalam evolusi galaksi dan pembentukan struktur skala besar. Selain itu, medan magnet berperan penting dalam dinamika gas pada awan molekul. Dengan medan magnet yang kuat, beberapa inti awan terbentuk dengan massa yang besar. Medan magnet juga mengontrol densitas dan perambatan berkas kosmik (\textit{cosmic rays}). Bersama dengan berkas kosmik, medan magnet dapat menghasilkan tekanan untuk mempercepat aliran gas panas, khususnya dalam galaksi dengan laju pembentukan bintang yang tinggi pada awal alam semesta. \\
	

\begin{table}[H] 
\begin{center}
\caption[Hasil \textit{Fitting} Kurva Rotasi M31 dengan dan tanpa Medan Magnet]{Hasil \textit{Fitting} Kurva Rotasi tanpa \textit{Dark Matter} (hanya Medan Magnet), dengan \textit{Dark Matter}, dan dengan Kontribusi Medan Magnet dan \textit{Dark Matter} untuk $r\geq 3$ kpc (sumber: Ruiz-Granados et al., 2012)}
\begin{tabular}{ccc}
\hline
Model (Parameter) & $\chi^{2}$ untuk \textit{Best-fit} & \textit{Reduced-}$\chi^{2}$ \\
\hline
MAG ($r_{1}$) & 4.37 & 0.34 \\
ISO tanpa MAG ($\rho_{0}, R_{h}$) & 8.81 & 0.68 \\
ISO dengan MAG ($r_{1}$, $\rho_{0}, R_{h}$) & 4.39 & 0.34 \\
\hline
\end{tabular}
\label{table:fitting}
\end{center}
\end{table} 