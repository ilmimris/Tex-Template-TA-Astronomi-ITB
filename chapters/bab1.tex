\chapter{PENDAHULUAN}
\vspace{1.0cm}

%-----------------------------------------------------------------------------%
\section{Latar Belakang}
%-----------------------------------------------------------------------------%

Lorem ipsum dolor sit amet, odio utroque definiebas ut quo, delenit omittam ne nec. Ius in assentior consectetuer, eos id malorum prodesset accommodare. Quot explicari definitionem eam eu, magna adipiscing eu nec. Dicta dicam sanctus vis cu, vel ne autem civibus facilisis. Ad dicat dolores pro, sea ex wisi justo possim, alienum reprehendunt vim ad. Sed quas verear ea, et wisi timeam percipitur his. Probo scaevola vim cu.\\

Ea vix assum recusabo, fabulas maiestatis ei sed. Pri wisi omnesque ex. Eum ea mundi laoreet appellantur, postea vidisse efficiantur sed ad. Duo case civibus ea, no pro recusabo scripserit. Has id audire deterruisset. Eam cu cibo exerci, et facilisis consetetur vix, mel an soleat ceteros.\\

Ex prima eirmod vulputate pri, eum no essent mandamus. Albucius accusamus salutatus vix at, assum ullamcorper ex sea. Vis eleifend consetetur ut, ex ius verear rationibus sadipscing. Meliore assentior sit cu. Vidisse omittantur vim ne. Regione accusam vituperatoribus vel ut, et sit commodo concludaturque.\\

%-----------------------------------------------------------------------------%	 
\section{Rumusan Masalah}
%-----------------------------------------------------------------------------%
Berdasarkan latar belakang yang telah diuraikan, dirumuskan masalah yang akan dibahas dalam Tugas Akhir ini sebagai berikut:
\begin{enumerate}
	\item Rumusan masalah pertama?
\pagebreak	
	\item Rumusan masalah kedua?
\end{enumerate}

%-----------------------------------------------------------------------------%
\section{Tujuan}
%-----------------------------------------------------------------------------%

Tujuan dari Tugas Akhir ini adalah untuk ...

%-----------------------------------------------------------------------------%
\section{Metode Penelitian dan Teknik Pengumpulan Data}
%-----------------------------------------------------------------------------%

Metode penelitian dalam Tugas Akhir ini adalah ..., yaitu:
\begin{enumerate}
	\item metode 1.
	\item metode 2. 
	\item metode 3.
\end{enumerate}
	
	Adapun teknik pengumpulan data yang digunakan dalam Tugas Akhir ini adalah ... \\

%-----------------------------------------------------------------------------%
\section{Sistematika Penulisan}
%-----------------------------------------------------------------------------%

	Tugas Akhir ini terdiri dari lima bab. Bab I merupakan bab pendahuluan yang terdiri atas latar belakang pemilihan topik Tugas Akhir, rumusan masalah, tujuan dari Tugas Akhir, metode penelitian dan teknik pengumpulan data yang digunakan dalam Tugas Akhir, serta sistematika dalam penulisan Tugas Akhir. Bab II membahas tentang teori dasar. Bab III membahas mengenai data yang digunakan dalam Tugas Akhir beserta metode pengolahannya. Bab IV membahas mengenai hasil dari pengolahan data beserta analisisnya. Bab V terdiri atas simpulan dan saran Penulis terhadap permasalahan yang dibahas dalam Tugas Akhir ini. \\
