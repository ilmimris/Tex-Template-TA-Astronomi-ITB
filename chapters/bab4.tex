\chapter{HASIL DAN ANALISIS}
\vspace{1.0cm}

%-----------------------------------------------------------------------------%
\section{ANALISIS 1}
%-----------------------------------------------------------------------------%
Lorem ipsum dolor sit amet, odio utroque definiebas ut quo, delenit omittam ne nec. Ius in assentior consectetuer, eos id malorum prodesset accommodare. Quot explicari definitionem eam eu, magna adipiscing eu nec. Dicta dicam sanctus vis cu, vel ne autem civibus facilisis. Ad dicat dolores pro, sea ex wisi justo possim, alienum reprehendunt vim ad. Sed quas verear ea, et wisi timeam percipitur his. Probo scaevola vim cu. \\

Ea vix assum recusabo, fabulas maiestatis ei sed. Pri wisi omnesque ex. Eum ea mundi laoreet appellantur, postea vidisse efficiantur sed ad. Duo case civibus ea, no pro recusabo scripserit. Has id audire deterruisset. Eam cu cibo exerci, et facilisis consetetur vix, mel an soleat ceteros.\\

Ex prima eirmod vulputate pri, eum no essent mandamus. Albucius accusamus salutatus vix at, assum ullamcorper ex sea. Vis eleifend consetetur ut, ex ius verear rationibus sadipscing. Meliore assentior sit cu. Vidisse omittantur vim ne. Regione accusam vituperatoribus vel ut, et sit commodo concludaturque.\\
\begin{itemize}
\item $(\frac{M}{L})_{disk}$ = 0,939 $\pm$ 0,006 $\frac{M_{\odot}}{L_{\odot}}$
\item $(\frac{M}{L})_{bulge}$ = 1,057 $\pm$ 0,00832 $\frac{M_{\odot}}{L_{\odot}}$
\item $R_{c}$ = 5,260 $\pm$ 0,011 kpc
\item $\rho_{h}(R_{c})$ = 0,045 $\pm$ 0,00003 $\frac{M_{\odot}}{pc^{3}}$
\item $\chi_{\nu}^{2}$ = 1,383 
\end{itemize}
\begin{figure}[H]
\centering 
\includegraphics[scale=0.6]{pics/dekom2841.png}
\caption[Plot Kurva Rotasi Galaksi NGC 2841 tanpa Medan Magnet]{Dekomposisi kurva rotasi NGC 2841 dengan tiga komponen galaksi, yaitu piringan bintang, gas, \textit{bulge} dan \textit{halo dark matter}.}
\label{fig:ngc2841}
\end{figure}

%-----------------------------------------------------------------------------%
\section{ANALISIS}
%-----------------------------------------------------------------------------%
\textit{Fitting} kurva rotasi NGC 7331 untuk kasus pertama, kedua dan ketiga dilakukan dengan metode yang sama seperti \textit{fitting} kurva rotasi NGC 2841 maupun NGC 6964. Gambar \ref{fig:ngc7331} menunjukkan hasil \textit{fitting} kurva rotasi untuk kasus pertama. \\
\begin{figure}[H]
\centering 
\includegraphics[scale=0.6]{pics/dekom7331.png}
\caption[Plot Kurva Rotasi NGC 7331 tanpa Medan Magnet]{Dekomposisi kurva rotasi NGC 7331 dengan menggunakan komponen piringan, gas, \textit{bulge}, dan \textit{halo dark matter}.}
\label{fig:ngc7331}
\end{figure}
Parameter-parameter yang dihasilkan dari \textit{fitting} kurva rotasi tersebut adalah sebagai berikut:
\begin{itemize}
\item $(\frac{M}{L})_{disk}$ = 0,698 $\pm$ 0,001 $\frac{M_{\odot}}{L_{\odot}}$
\item $(\frac{M}{L})_{bulge}$ = 0,702 $\pm$ 0,002 $\frac{M_{\odot}}{L_{\odot}}$
\item $R_{c}$ = 19,820 $\pm$ 0,099 kpc
\item $\rho_{h}(R_{c})$ = 0,005 $\pm$ 0,00001 $\frac{M_{\odot}}{pc^{3}}$
\item $\chi_{\nu}^{2}$ = 3,357
\end{itemize}
	Seperti halnya dengan NGC 2841 dan NGC 6946, \textit{fitting} kurva rotasi pada kasus kedua diawali dengan melakukan \textit{fitting} profil medan magnet dalam arah azimuthal yang menghasilkan parameter $B_{1}$ dan $r_{1}$. Gambar \ref{fig:fitngc7331} menunjukkan hasil fitting profil medan magnet dalam arah azimuthal.\\
\begin{figure}[H]
\centering 
\includegraphics[scale=0.6]{pics/fitting7331.png}
\caption[Plot \textit{fitting} Profil Medan Magnet Azimuthal Galaksi NGC 7331]{Hasil \textit{fitting} profil medan magnet dalam arah azimuthal untuk NGC 7331.}
\label{fig:fitngc7331}
\end{figure}
	Berdasarkan \textit{fitting} profil medan magnet, didapatkan nilai $B_{1}$ sebesar 1002,05 $\pm$ 0,24 $\mu$G dan nilai  sebesar 0,14 $\pm$ 0,0005 kpc dengan \textit{reduced chi-square} sebesar 12,78. Terlihat bahwa \textit{fitting} dengan fungsi tunggal kurang memberikan hasil yang baik pada radius menengah, namun hasil ini tetap dipakai karena terutama kita lebih tertarik pada kontribusi medan magnet di daerah luar. Selanjutnya dilakukan \textit{fitting} kurva rotasi yang ditunjukkan dalam Gambar \ref{fig:ngc7331b} berikut.\\
\begin{figure}[H]
\centering 
\includegraphics[scale=0.6]{pics/dekom7331b.png}
\caption[Plot Kurva Rotasi Galaksi NGC 7331 dengan Medan Magnet yang Difiksasi]{Dekomposisi kurva rotasi NGC 7331 untuk kasus kedua: kontribusi komponen piringan, gas, \textit{bulge}, dan medan magnet dibuat tetap (\textit{fix}) dan \textit{halo dark matter} sebagai komponen yang divariasikan.}
\label{fig:ngc7331b}
\end{figure}
Parameter yang dihasilkan dari \textit{fitting} kurva rotasi tersebut adalah:
\begin{itemize}
\item $R_{c}$ = 13,379 $\pm$ 0,115 kpc
\item $\rho_{h}(R_{c})$ = 0,007 $\pm$ 0,00007 $\frac{M_{\odot}}{pc^{3}}$
\item $\chi_{\nu}^{2}$ = 2,532
\end{itemize}
	Sedangkan untuk hasil \textit{fitting} kasus ketiga dengan menggunakan komponen piringan, gas, \textit{bulge} dengan \textit{halo dark matter} dan medan magnet yang dibebaskan dalam \textit{fitting} ditunjukkan oleh Gambar \ref{fig:ngc7331c} berikut.\\
\begin{figure}[H]
\centering 
\includegraphics[scale=0.6]{pics/dekom7331c.png}
\caption[Plot Kurva Rotasi Galaksi NGC 7331 dengan Medan Magnet yang Dibebaskan]{Dekomposisi kurva rotasi NGC 7331 untuk kasus ketiga: kontribusi komponen piringan, gas, \textit{bulge} dibuat tetap (\textit{fix}), sedangkan medan magnet dan \textit{halo 
dark matter} sebagai komponen yang divariasikan.}
\label{fig:ngc7331c}
\end{figure}
Parameter yang dihasilkan sebagai berikut,
\begin{itemize}
\item $R_{c}$ = 13,631 $\pm$ 0,154 kpc
\item $\rho_{h}(R_{c})$ = 0,007 $\pm$ 0,0001 $\frac{M_{\odot}}{pc^{3}}$
\item $B_{1}$ = 980,838 $\pm$ 0,354 $\mu$G
\item $r_{1}$ = 0,0295 $\pm$ 0,0056 kpc
\item $\chi_{\nu}^{2}$ = 2,445 
\end{itemize}
	Sama seperti kurva rotasi NGC 6946, kurva rotasi NGC 7331 menghasilkan \textit{fitting} yang lebih baik apabila ditambahkan kontribusi medan magnet.  Hal tersebut dapat dilihat dari nilai \textit{reduced chi-square} yang lebih kecil mendekati nilai satu. Gambar \ref{fig:kurva7331} menunjukkan perbandingan kurva rotasi dari ketiga kasus.\\
\begin{figure}[H]
\centering 
\includegraphics[scale=0.6]{pics/kurva7331.png}
\caption[Plot Perbandingan Kurva Rotasi NGC 7331 dengan Tiga Kasus.]{Perbandingan kurva rotasi NGC 7331 dari ketiga kasus yang ditinjau. }
\label{fig:kurva7331}
\end{figure}

Tabel \ref{table:fit} berikut meringkaskan nilai parameter-parameter hasil \textit{fitting} dari ketiga kasus untuk semua galaksi yang ditinjau.\\
\begin{center}
\begin{longtable}[t]{ccccc}

\caption[Parameter Hasil \textit{Fitting} Kurva Rotasi]{Parameter Hasil \textit{Fitting}}\\

\hline
Galaksi & Kasus & Parameter & Nilai & $\chi_{\nu}^{2}$ \\
\hline
\endhead

\multirow{10}{*}{NGC 2841} & \multirow{4}{*}{I} & $(\frac{M}{L})_{disk}$ & 0,939 $\pm$ 0,006 & \multirow{5}{*}{1.383} &\\
& & $(\frac{M}{L})_{bulge}$ & 1,057 $\pm$ 0,000832 \\
& & $R_{c}$ & 5,260 $\pm$ 0,011 \\
& & $\rho_{h}(R_{c})$ & 0,045 $\pm$ 0,00003 \\ \cline{2-5}
\pagebreak
& \multirow{2}{*}{II} & $R_{c}$ & 5,248 $\pm$ 0,010 & \multirow{2}{*}{1.380} &\\
& & $\rho_{h}(R_{c})$ & 0,046 $\pm$ 0,00010 \\ \cline{2-5}
& \multirow{4}{*}{III} & $R_{c}$ & 5,100 $\pm$ 0,013 & \multirow{4}{*}{1.108} &\\
& & $\rho_{h}(R_{c})$ & 0,046 $\pm$ 0,00010 \\
& & $B_{1}$ & 476,204 $\pm$ 0,033 \\
& & $r_{1}$ & 0,819 $\pm$ 0,0256 \\ \hline
\multirow{10}{*}{NGC 6949} & \multirow{4}{*}{I} & $(\frac{M}{L})_{disk}$ & 0,677 $\pm$ 0,002 & \multirow{5}{*}{1.466} &\\
& & $(\frac{M}{L})_{bulge}$ & 0,761 $\pm$ 0,005 \\
& & $R_{c}$ & 2,691 $\pm$ 0,012 \\
& & $\rho_{h}(R_{c})$ & 0,056 $\pm$ 0,00001 \\ \cline{2-5}
& \multirow{2}{*}{II} & $R_{c}$ & 2,724 $\pm$ 0,006 & \multirow{2}{*}{3,408} &\\
& & $\rho_{h}(R_{c})$ & 0,054 $\pm$ 0,00004 \\ \cline{2-5}
& \multirow{4}{*}{III} & $R_{c}$ & 2,703 $\pm$ 0,005 & \multirow{4}{*}{1.466} &\\
& & $\rho_{h}(R_{c})$ & 0,055 $\pm$ 0,00004 \\
& & $B_{1}$ & 462,282 $\pm$ 0,0042 \\
& & $r_{1}$ & 0,0064 $\pm$ 0,0042 \\ \hline
\multirow{10}{*}{NGC 7331} & \multirow{4}{*}{I} & $(\frac{M}{L})_{disk}$ & 0,698 $\pm$ 0,001 & \multirow{5}{*}{3,357} &\\
& & $(\frac{M}{L})_{bulge}$ & 0,702 $\pm$ 0,002 \\
& & $R_{c}$ & 19,820 $\pm$ 0,099 \\
& & $\rho_{h}(R_{c})$ & 0,005 $\pm$ 0,00001 \\ \cline{2-5}
& \multirow{2}{*}{II} & $R_{c}$ & 13,379 $\pm$ 0,115 & \multirow{2}{*}{2,532} &\\
& & $\rho_{h}(R_{c})$ & 0,007 $\pm$ 0,0001 \\ \cline{2-5}
& \multirow{4}{*}{III} & $R_{c}$ & 13,631 $\pm$ 0,154 & \multirow{4}{*}{2,445} &\\
& & $\rho_{h}(R_{c})$ & 0,007 $\pm$ 0,0001 \\
& & $B_{1}$ & 980,838 $\pm$ 0,0456 \\
& & $r_{1}$ & 0,0295 $\pm$ 0,0456 \\ \hline
\multirow{10}{*}{M 31} & \multirow{4}{*}{I} & $(\frac{M}{L})_{disk}$ & 1,027 $\pm$ 0,025 & \multirow{5}{*}{1,146} &\\
& & $(\frac{M}{L})_{bulge}$ & 0,217 $\pm$ 0,0003 \\
& & $R_{c}$ & 2,378 $\pm$ 0,042 \\
& & $\rho_{h}(R_{c})$ & 0,096 $\pm$ 0,0005 \\ \cline{2-5}
& \multirow{2}{*}{II} & $R_{c}$ & 2,464 $\pm$ 0,012 & \multirow{2}{*}{1,148} &\\
& & $\rho_{h}(R_{c})$ & 0,090 $\pm$ 0,00003 \\ \cline{2-5}
& \multirow{4}{*}{III} & $R_{c}$ & 2,343 $\pm$ 0,013 & \multirow{4}{*}{1,036} &\\
& & $\rho_{h}(R_{c})$ & 0,091 $\pm$ 0,0001 \\
& & $r_{1}$ & 320,155 $\pm$ 15,650 \\ 
\hline
\label{table:fit}
\end{longtable}
\end{center}